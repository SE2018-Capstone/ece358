\documentclass[12pt]{article}
\usepackage{amsmath,amsfonts}
\usepackage{graphicx}
\usepackage{enumerate}
\usepackage{algpseudocode}
\usepackage{tikz}
\renewcommand{\thesection}{Problem \arabic{section}}
\renewcommand{\thesubsection}{(\alph{subsection})}
\usepackage{fullpage,url,amssymb,epsfig,color,xspace}

\newcommand{\class}{ECE 358}
\newcommand{\subtitle}{Pencil-n-Paper Assignment 1}

\usepackage[pdftitle={\class \subtitle},%
pdfsubject={University of Waterloo, \class, Spring 2017},%
pdfauthor={Shiranka Miskin}]{hyperref}

%this marks the beginning of the document. Everything before this is called the Preamble.
\begin{document}
\begin{center}
{\Large\bf University of Waterloo}\\
\vspace{3mm}
{\Large\bf \class, Spring 2017}\\
\vspace{2mm}
{\Large\bf \subtitle}\\
\vspace{3mm}
\textbf{Shiranka Miskin, Dhruv Lal, Sam Maier}
\end{center}

\section{}
\subsection{}
If the network is not store-n-forward, $d_{\text{end-to-end}} = \frac{L}{R}$, as
incoming bits can immediately flow to the next router and there is no
propagation delay.

\subsection{}
Given $N - 1$ routers between two hosts, a bit from the source must travel
across $N$ links.  Each link introduces a $d_{\text{end-to-end}}$ of
$\frac{L}{R}$ in order for all $L$ bits to be stored in the next router before
it is forwarde again.  Since there is no propagation, queuing, or processing
delay, this accounts for the entire delay, given $N$ routers there is a total delay of
$d_{end-to-end} = N\frac{L}{R}$.


\subsection{}
The delay is $N\frac{L}{R} + (P - 1)\frac{L}{R}$, since once one router has
forwarded one packet it may immediately receive another without waiting for the
first packet to reach the destination.  The first packet arrives at time
$N\frac{L}{R}$, the next packet arrives at time $N\frac{L}{R} + \frac{L}{R}$,
and each successive packet arrives $\frac{L}{R}$ after the previous one.

\subsection{}
Since every packet but the first must now wait an extra $\frac{L}{R}$ for the
last bit of the previous packet to be sent from the next router, the delay is
now $N\frac{L}{R} + (2P - 2)\frac{L}{R}$.


\section{}
\begin{verbatim}
Construct e where e(u,v) = G.c(u,v) - G.a(u,v)
Construct G' = <V, e>
Use a standard shortest path algorithm such as Dijkstra's algorithm or
Floyd-Warshall on G' and return the path if it exists

return 'none'
\end{verbatim}

\section{}
\subsection{}
$$d_{\text{prop}} = \frac{m}{s}$$
\subsection{}
$$d_{\text{trans}} = \frac{L}{R}$$
\subsection{}
$$d_{\text{end-to-end}} = \frac{L}{R} + \frac{m}{s}$$
\subsection{}
The first bit of the packet left $A$ at time 0, therefore at time $t =
d_{\text{trans}}$ it will have travelled $d_{\text{trans}} \cdot s$ meters along
the link.

\section{}
Each packet takes $\frac{L}{R}$ to completely exit through the link, therefore
the first packet in the queue will have a queueing delay of $0$, the second will
have a delay of $\frac{L}{R}$, the third will have $2\frac{L}{R}$, and so on
until the $N$th packet which has a queueing delay of $(N - 1)\frac{L}{R}$.  This
means the average queuenig delay is 
$$\frac{\sum_{x = 0}^{N - 1} x\cdot\frac{L}{R}}{N} = \frac{L}{2NR}(N - 1)(N) =
\frac{L(N - 1)}{2R}$$

\section{}
If $a \leq \mu$, then there will be no queueing delay, and $d_{\text{total}} =
d_\text{trans} = \frac{n}{\mu}$.  If $a > \mu$, then the last packet will arrive
at time $\frac{n - 1}{a}$, and will depart at time $\frac{n - 1}{\mu}$, therefore there
will be a queueing delay of $\frac{n - 1}{\mu} - \frac{n - 1}{a}$.  That puts
the total delay when $a > \mu$ at $\frac{2n - 1}{\mu} - \frac{n -1}{a}$.

\section{}

40 terabytes is 320,000,000,000,000 bits, therefore it will take
$\frac{320 \times 10^{12}}{100\times10^{6}} = 3.2\times10^{6}\text{ seconds} =
53333.33\text{ minutes} = 888.89\text{ hours} = 37.04\text{ days}$ to
transfer the complete file.  Overnight delivery is definitely the superior option.

\section{}

From 1c we determined that for $P$ packets, $d_{\text{end-to-end}} =
(N + P - 1)\frac{L}{R}$.  In this problem, $N = 3$, $P = \frac{F}{S}$, and $L = S +
80$.  This gives us

$$
\begin{aligned}
    d_{\text{end-to-end}} &= \left(\frac{F}{S} + 2\right)\frac{S + 80}{R}\\
    &= \frac{(S + 80)(F + 2S)}{RS}\\
    \frac{\partial d_{\text{end-to-end}}}{\partial S} &= \frac{2(-40F + S^2)}{RS^2}\\
    \frac{2(-40F + S^2)}{RS^2} &= 0 \Rightarrow S = \sqrt{40F}\\
\end{aligned}
$$

Therefore to minimize the delay from $A$ to $B$ we should use packets of size $S = \sqrt{40F}$












\end{document}
