\documentclass[12pt]{article}
\usepackage{amsmath,amsfonts}
\usepackage{graphicx}
\usepackage{enumerate}
\usepackage{algpseudocode}
\usepackage{tikz}
\renewcommand{\thesection}{Problem \arabic{section}}
\renewcommand{\thesubsection}{(\alph{subsection})}
\usepackage{fullpage,url,amssymb,epsfig,color,xspace}

\newcommand{\class}{ECE 358}
\newcommand{\subtitle}{Pencil-n-Paper Assignment 2}

\usepackage[pdftitle={\class \subtitle},%
pdfsubject={University of Waterloo, \class, Spring 2017},%
pdfauthor={Shiranka Miskin}]{hyperref}

%this marks the beginning of the document. Everything before this is called the Preamble.
\begin{document}
\begin{center}
{\Large\bf University of Waterloo}\\
\vspace{3mm}
{\Large\bf \class, Spring 2017}\\
\vspace{2mm}
{\Large\bf \subtitle}\\
\vspace{3mm}
\textbf{Shiranka Miskin, Dhruv Lal, Sam Maier}
\end{center}

\section{}
\subsection{}
The two would be equivalent, since both would require the entire link bandwidth.
In parallel it would take $3\frac{200}{150} + \frac{100000}{150} +
(3\frac{200}{\frac{150}{10}} + \frac{100000}{\frac{150}{10}}) = 7377.33$
seconds, and serially it would take $3\frac{200}{150} + \frac{100000}{150} +
10(3\frac{200}{150} + \frac{100000}{150}) = 7377.33$ seconds.

\subsection{}
Persistent provides minor gains, as a TCP connection would have to be
established only once for all 11 transfers, rather than 11 times, saving 2 RTT
delays for each TCP handshake.  Compared to the thousands of seconds that the
entire transaction requires, this gain is insignificant.

\subsection{}
Bob will have an advantage, as his effective link bandwidth will be much
greater than that of his peers.  Each one of his many parallel connections receives
$\frac{1}{14}$ of the bandwidth, while his peers get a total of $\frac{1}{14}$
of the bandwidth each.

\subsection{}
Bob will no longer have an advantage as now the other users are in the exact
same situation as himself, using parallel non-persistent HTTP.

\section{}
\subsection{}
By following aliases, \texttt{www.ibm.com} eventually resolves to
\texttt{96.16.43.195}.  \texttt{www.ibm.com} is not the actual host name of
\texttt{96.16.43.195}, which is what \texttt{-x} returns.
\subsection{}

\begin{tabular}{|p{5cm}|c|p{5cm}|}
    \hline
    \textbf{Name} & \textbf{Type} & \textbf{Value}\\
    \hline
    www.ibm.com & CNAME & www.ibm.com.cs186.net\\
    cs186.net & NS & ns.events.ihost.com\\
    www.ibm.com.cs186.net & CNAME & www2.ibm.com.edgekey.net\\
    www2.ibm.com.edgekey.net & CNAME & www2.ibm.com.edgekey.net .globalredir.akadns.net \\
    akadns.net & NS & a18-128.akadns.org\\
    www2.ibm.com.edgekey.net .globalredir.akadns.net & CNAME & e2874.dscx.akamaiedge.net\\
    dscx.akamaiedge.net & NS & n0dscx.akamaiedge.net\\
    e2874.dscx.akamaiedge.net & A & 96.16.43.195\\
    \hline
\end{tabular}

\section{}
\subsection{}

If the server uploads the file to all of its peers at the same time at a rate of
$\frac{u_s}{N}$, and $d_{\text{min}} \geq \frac{u_s}{N}$, since the server is
the bottleneck, the distribution time will be $\frac{NF}{u_s}$, the time it takes
to send the total $NF$ bits across from the server.

\subsection{}
If $\frac{u_s}{N} \geq d_{\text{min}}$, then if the server distributes to all of
its peers at the same time at a rate of $d_{min}$, the distribution time will be
$\frac{F}{d_\text{min}}$.

\subsection{}
Each peer could be connected to every other peer, and each peer $p_i$ could
transmit parts of the data it has to all other peers at a rate of $u_i/N$.  The
total rate of data being transferred through the network is $u_s + u_1 + \cdots
+ u_N$, therefore a given peer will be receiving data at a rate of $\frac{u_s +
u_1 + \cdots + u_N}{N}$ on average.  The server wouldn't send the same bits to
every client, that way each client receives data that the other clients do not
have, and they can begin distributing that.  Since $u_s < \frac{u_s + u_1 +
\cdots + u_N}{N}$, the server, which must send all $F$ of its bits to distribute
the entire file over the network, will be the bottleneck, and the distribution
time will be $\frac{F}{u_s}$.

\subsection{}
With the same setup as (c), if $u_s \geq \frac{u_s + u_1 + \cdots + u_N}{N}$,
then $u_s$ is no longer the bottleneck and the distribution time is instead
dependant on the average transfer rate.  A given $u_i, i \neq s$ will not
bottleneck the network as it need not send all $F$ bits.  A peer which
uploads at half the rate of the average could be compensated for by a peer which
uploads at three halfs of the average.  The distribution time will therefore be
$\frac{F}{(u_s + u_1 + \cdots + u_N)/N}$, which can be written as $NF/(u_s+u_1+
\cdots+u_N)$.

\section{}

Let $D(k)$ be the difference between a key $k$ and its nearest neighbor
$$E[k] = \sum_{i = 0}^8 i \cdot \text{Pr}\{D(k) = i\}$$

$$\begin{aligned}
    \text{Pr}\{D(k) = 0\} &= 1 - \text{Pr}\{\text{All other points are different values than $k$}\}\\
    &= 1 - \left(\frac{15}{16}\right)^4 \text{(15 valid positions left to choose from)}\\
    \\
    \text{Pr}\{D(k) = i\text{, $i \in [1, 7]$}\} &= \text{Pr}\{\text{All other points are $i$ or more away from $k$}\}\\ 
                                                 &- \text{Pr}\{\text{All other points are $i + 1$ or more away from $k$}\}\\
    &= 1 - \left(\frac{2(i - 1) + 1}{16}\right)^4 - 1 - \left(\frac{2i + 1}{16}\right)^4\\ 
    \text{Pr}\{D(k) = 8\} &= \text{Pr}\{\text{All other points are at the furthest point from $k$}\}\\
    &= \left(\frac{1}{16}\right)^4 \text{(Only one valid position to choose from)}\\
\end{aligned}
$$

$$0\cdot(1 - \frac{15}{16}^4) + 8\cdot(\frac{1}{16})^4 + \sum_{i = 1}^7 i \left(\left(\frac{15 - 2(i - 1)}{16}\right)^4
- \left(\frac{15 - 2i}{16}\right)^4\right) \approx 1.57922$$


\section{}
\subsection{}
\begin{verbatim}
lookup(k):
    if (itsMe(k)) return p

    next_peer = FT_p[m]
    for i = 2 .. m:
        if FT[i] >= k:
            next_peer = FT_p[i - 1]
            break

   return next_peer 
\end{verbatim}


\subsection{}
Each hop travels at least half the arclength from a starting node $p$ and the
peer $r$ where $r$ is the peer immediately before $\text{succ}(k)$.  Since the
peers are evenly distributed along the domain, travelling at least half the arc
length will also result in travelling at least half the number of peers between
$p$ and $r$.  This gives us the upper bound on the worst case number of hops
being $O(\log n)$ hops, where $n$ is the number of peers.  We can prove that
this is a tight bound by constructing the case where $r$ is the immediate
predecessor of $p$.  $r$ will only be hopped to once $p$ appears in the finger
table of a peer that is calling \texttt{lookup}, and since each hop can at most
halve the distance around the circle, it will take $O(\log n)$ hops to finally
be able to check $p$ again.
\end{document}
